\section{Related Work}

Multiple apporaches have been proposed to solve the problem of user authentication through keystroke dynamics. It is possible to classify said approaches into two main categories: those that focus on what data is obtained and those that focus on the classification algorithm. In the first category, work have been done comparing and analyzing results with free text and transcribed text \cite{old,programmers} and with memorized text like usernames and passwords \cite{bleha1990login, pnn}. In the second category, work have been done using different classification algorithms, such as distance metrics \cite{combine_distance}, multiple machine learning techniques \cite{machine_learning} and Neural Networks \cite{deep_learning, maiorana2019deepkey}.

Previos work has shown that in some cases simple distance based clustering algorithms can acomplish good performance \cite{robinson1998login, bergadano2002norm_distance,de2000mahalanobis,programmers} this have been explained by the fact that the there are some distances that handle effectively characteristics of the features present in keystroke dynamics, like the correlations  and good interaction between the features. Some examples of these distances are Mahalanobis which takes into account the covariance matrix of the features for better differentiation and  Manhattan distance which is more robust, it is not affected by the scale of the features \cite{combine_distance}. 

It also have been shown that the selection of the features is a key factor for the performance of the models and that the algoritms used for the classification are less important \cite{pnn}. This can be seen in the continious integration of new features thoughout the development of the techniques. At the beginning, the data was obtained from the time between 2 keystrokes  \cite{bleha1990login} and then the pressing time was added \cite{robinson1998login}. Later more sofisticated features were added like the edit time \cite{machine_learning}, normalized data so that the emotional state of the user does not affect the results  and the introduction of thrigraphs \cite{bergadano2002norm_distance}.

In recent years the use of deep learning techniques has been studied and proposed to solve the problem of user biometric recognition through keystroke dynamics. A hybrid model involving a CNN and a RNN was proposed in \cite{deep_learning} using feature transformation to adapt the input data to a image-like format for better results. More simple models have also been proposed \cite{maiorana2019deepkey}, using CNN an fixing the input size with a 6 digit pin on mobile devices.