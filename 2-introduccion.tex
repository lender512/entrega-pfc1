\section{Introduction}

Since the beginning of the digital age, cybersecurity has been developing and adapting to 
the new threats that arise in the digital world. One of the key topics of great importance 
in this field is user authentication. Different methods have been used over the years: passwords, iris recognition, fingerprint matching, etc. However, most of these methods seek to guarantee user authentication only at the beginning of the session of interaction between a 
human and a device, but not through it. This is a problem, since if a user leaves their 
session open, anyone can access their information. For this reason, the use of behavioral bio-metrics has been proposed, which allows the user to be authenticated not only at the begining of the session but also throughout it \cite{behavior}. One of the behavioral bio-metrics applications is called Keystroke Dynamics, which consists of the authentication a user at the begining abd 
throughout a session by analyzing the patterns and rhythm that they present when writing using a keyboard.


The main problem with the Keystroke Dynamics technique as a method for user authentication 
is the different classification algorithms used to cluster users. Different comparison 
techniques have been used though the years, distance metrics \cite{combine_distance}, 
multiple machine learning techniques \cite{machine_learning} and Neural Networks 
\cite{pnn, deep_learning}. All these techniques have very different results, some of the best achieve 
an average false rate of 2\% while others are between 8 and 27\% \cite{typing_patterns}. 
In addition, it has been shown that precision can vary depending on what is being written. 
It has been possible to observe how the precision goes from 79\% with text that is transcribed 
to 21\% with text that is freely written \cite{old} and that if a person is writing code the 
algorithms can reach a precision of 98.6\% \cite{programmers}.

The main objective of this work is to implement two algorithms for user 
authentication through Keystroke Dynamics that have already been proposed. It seeks to 
compare the results obtained by each of the algorithms, and determine which of them is 
the most appropriate for this type of authentication. In addition, it seeks to determine 
the behavior of each of the algorithms with respect to the type of data that is entered, 
that is, if the algorithm is more accurate with free text or with text that is transcribed.